\documentclass[letterpaper,twocolumn,amsmath,amssymb,pre]{revtex4-1}
\usepackage{graphicx}% Include figure files
\usepackage{dcolumn}% Align table columns on decimal point
\usepackage{bm}% bold math
\usepackage{color}
\usepackage{breqn}

\newcommand{\red}[1]{{\bf \color{red} #1}}
\newcommand{\blue}[1]{{\bf \color{blue} #1}}
\newcommand{\green}[1]{{\bf \color{green} #1}}
\newcommand{\rr}{\textbf{r}}
\newcommand{\refnote}{\red{[ref]}}

\newcommand{\fixme}[1]{\red{[#1]}}

%\newcommand{\derivation}[1]{#1} % Use this to show all derivations in detail
\newcommand{\derivation}[1]{} % Use this for nice pegagogical paper...

% needsworklater is used to annotate bits that need work, but that we
% can postpone for a while.
\newcommand{\needsworklater}[1]{\emph{[#1]}}
% needsworknow is intended to prioritize stuff that needs fixing.
\newcommand{\needsworknow}[1]{\textcolor{red}{[\emph{#1}]}}

\begin{document}
\title{E Coli project paper}

%\pacs{61.20.Ne, 61.20.Gy, 61.20.Ja}
%%%%%%%%%%%%%%%%%%%%%%%%%%%%%%%%%%%%%%%%%%%%%%%%%%%%%%%%%%%%
\begin{abstract}
  This paper is about science.
\end{abstract}

\section{Introduction}
\subsection{What is the MinD system and why is it important?}
-Systm of proteins in E.Coli and other cells.
-Theorized to be instrumental in cell citokenisis. Reference experiments
\subsection{How proteins move in cell}
-Reference experimental showing proteins oscillating
-Reference theory showing difEQ model shows oscillations
-Reference Mannik shoving into crevices.
-Worthwhile studying effect of walls shape on the movement of cells
(Sign post of what to expect from this paper)

\section{Methods and Initial Conditions}

\section{Specific Results}
-Data and plots that show concrete results.

\section{Interpretation of Data}
-Discussion of conceptual reasons of why we see what we see
-Plots that are more interpretive (area-rating)
-Some sort of predictive claim?

\section{Conclusion}


\appendix

\section*{Appendix}


\end{document}
